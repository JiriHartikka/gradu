\section{Background}

As the mobile devices become more and more essential for our every day lives, the need for longer and longer battery lives increases. Despite the impact that mobile device battery life has on every day lives of hundreds of millions of people worldwide, the factors which affect a mobile device's battery life have not been studied extensively.        

\subsection{Mobile Device Battery Life}
The impact of using 2G and 3G networks for the mobile phones battery life was studied in \cite{5357972}. The authors used a Nokia N95 phone to test the relative battery consumption of various tasks comparing the results of using GMS, a 2G mobile networking technology and UMTS, a 3G mobile networking technology. The tasks for which the battery consumption was measured included 

\begin{itemize}
\item Sending 50 SMS messages of 100 bytes
\item Downloading 100 megabytes of data
\item Performing a 5 hour voice call 
\end{itemize} 

The conclusion of the study was that different networking technologies are energy efficient in different tasks. While the UMTS network was much more energy efficient for downloading data, the GMS was more energy efficient when sending SMS messages or performing voice calls. The authors argue that this information could be used to minimize energy consumption of mobile phones when multiple networking technologies are available.      

\subsection{Mobile Data Analysis}
 * System Settings and Variables

\subsection{Association Analysis}

