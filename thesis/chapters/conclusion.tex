\section{Conclusion}

This thesis work has presented a method for generating association rules from Carat dataset in order to estimate how mobile device system settings and context factors impact the level of energy consumption of a mobile device when using a particular mobile application. These association rules reveal non-trivial and perhaps unexpected connections between these settings and context factor and the level of energy consumption within the context of multiple mobile applications. For some reason, the generated association rules that predict low levels of energy consumption, seem to have much higher confidence than the rules which predict high levels energy consumption. This may be due to various reasons. One reason might be, that while the association analysis seems to be able to capture at least some circumstances which consistently lead to low energy consumption, the system settings and context variables available within the Carat dataset are inadequate for explaining unusually high energy consumption levels. It could even be, that the users whose devices have high energy consumption are generally running multiple mobile applications at the same time, which would naturally generate more noise to data points coming from those users. One could potentially test this hypothesis by adding the number of running applications to the list of variables from which the association rules are generated from. If this was the case, then one would expect to see rules where high number of running applications predicts high energy consumption. 

Another goal of thesis work was to implement a web based interface, so that users could search these association rules easily. The implementation has two web servers that communicate to one another using a simple HTTP based API. The back end of the service resides on a Spark cluster where it can execute the analysis engine on user supplied parameters as requested. This way the data analysis can be wrapped inside a single exchange of HTTP request and response. The front end of the service handles all things related to the graphical user interface: rendering the search form, fetching the rules from the back end and rendering the results. The front end of the service can reside wherever as long as the service back end can be reached by HTTP. This two-tier architecture allows the remote use of computational resources of a Spark cluster without exposing the Spark cluster environment to potential security vulnerabilities that a globally accessible web server might impose.

The implementations of both the data analysis and the user interface could be further improved. First of all, due to performance reasons, the data set had to be limited to around 16 GB, which is more than an order of magnitude less than the whole amount of available data. It is quite possible that the association analysis might reveal more fine grained dependencies between the context factors and system settings and the energy consumption of a device, if the analysis was performed using more of the available data. Different discretization strategies for the data might also affect the generated rules. Discretization of most numerical variables was done using quite an arbitrary number of percentiles, namely four. The implementation could easily be extended to allow the user to specify the number percentiles used in the discretization. 

The user interface could be improved in multiple ways. The user interface does not show the units of measurement nor does it show values of the break points of the percentiles, which could give the user a clearer sense of how a certain value range of a variable compares to the average value of the variable. Rendering of the rules could also be improved. Paging of the rules should definitely be implemented because browsing through as many as thousands of rules in a single page is cumbersome. The user could also benefit from a searching and filtering functionality in the front end of the service to be able quickly find the rules that the user considers interesting. 

This thesis work shows that the association analysis can effectively be applied to the domain of mobile device energy consumption modelling. This work also summarises the theoretical background of the state of the art methods used in association analysis and in the MapReduce programming model. The performance evaluation aspect of the association rule generation process is also discussed and results of the evaluation are presented. Additional work is still needed to optimize the performance of the analysis engine and the usability of the user interface and to find out how different data discretization approaches affect the generated association rules. 