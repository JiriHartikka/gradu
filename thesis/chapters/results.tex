\section{Results}

The purpose of this thesis work was to design and implement an interactive and semi automated web based tool for discovering association rules from the Carat data that indicate what system settings and usage patterns of a mobile application lead to increased battery consumption. In practise, this has proven to be quite challenging for a number of reasons:

\begin{itemize}
  \item Automatically deciding a sufficient threshold for support and confidence for generating rules is difficult 
  \item Number of generated rules increases rapidly as support and confidence thresholds are lowered
  \item Identifying interesting or relevant rules in presence of hundreds or thousands of rules is cumbersome
  \item Deciding how to discretize ordinal and continuous variables is non-trivial          
\end{itemize}

We will now look at the results of this work in two parts. In the first part we will be looking at the performance of the application and how it affects its usability. In the second part we will take a look at some example use cases of the system.  

\subsection{Performance Evaluation}

In order to understand the relationship between the number of generated rules and minimum support and confidence thresholds, a series of measurements were conducted on the Carat API prototype server. Figure~\ref{figure:number-of-ruless} shows the relationship of these variables on two selected applications, namely Spotify and Facebook mobile applications. The blue dots represent individual measurements. A minimum confidence threshold range of 0.3 to 0.9 and a minimum support threshold range of 0.0001 to 0.005 were both divided evenly by 10 points creating a grid of 100 points where the measurements were taken. The number of rules -axis is in $log_{10}$ scale to better illustrate the varying magnitudes of the number of generated rules. The transparent red plane was fitted to the measured points using the least squares method.

The number generated rules seems to grow exponentially on both axes when approaching zero, as can be seen by how well the measurements align with the least squares plane. This explosion in the number of generated rules makes it difficult for the user to extract useful rules from the system when small values for the thresholds are used. To mitigate this problem, the system provides two features:

\begin{itemize}
	\item The generated rules are sorted in the ascending order of their confidence, giving the more reliable rules a greater priority.    
         
	\item Attributes can be excluded from the analysis - potentially greatly reducing the number of generated rules. 
\end{itemize}        

\begin{figure}[htp]

\subfloat[Facebook mobile application]{%
  \includegraphics[width=\textwidth]{images/results/facebook_num_rules.png}%
}

\subfloat[Spotify mobile application]{%
  \includegraphics[width=\textwidth]{images/results/spotify_num_rules.png}%
}

\caption{Number of generated rules plotted as a function of minimum support threshold and minimum confidence threshold}
\label{figure:number-of-ruless}
\end{figure}

In addition to the number of generated rules, another metric that tells about the usability of the system is the time taken to generate the association rules. To measure the time of the rule generation as a function of minimum support threshold and minimum confidence threshold, a similar set up as with the number of generated rules was used. Figure~\ref{figure:analysis-runtime} shows these measurements. Like before, the blue dots represent the 100 measurement points and the transparent red plane represents a plane that was fitted to the points using the least squares method. The rule generation time increases as either axis approaches zero. The deviance is not huge however, as all the measured run times fall between 160 and 260 seconds. While this is a notable difference from the users perspective, the system remains usable even when the number of generated rules is in the order of $10^5$.   

\begin{figure}[htp]

\subfloat[Facebook mobile application]{%
  \includegraphics[width=\textwidth]{images/results/facebook_runtimes.png}%
}

\subfloat[Spotify mobile application]{%
  \includegraphics[width=\textwidth]{images/results/spotify_runtimes.png}%
}
 
\caption{Association rule generation time as a function of minimum support threshold and minimum confidence threshold}
\label{figure:analysis-runtime}
\end{figure}  

\subsection{Overview on Generated Rules}